\chapter{Conclusions and Future Work}
A multi-phase species transport model was successfully implemented into CTF along side a simplified interfacial area tracking method. This model is meant to help inform users fission product transport and to aid in design optimization. In this report only xenon and iodine were tested but the model can handle up to any number of fission products of interest. All of the nuclear source terms studied were assumed to be held constant however, in a true MSR some will vary. Generation rates from fission will change as a function of temperature and salt composition. Coupling species transport into VERA will inform neutronics codes such as MPACT on spacial material composition. Along with nuclear source terms, variables such as mass transfer coefficient and Henry's law constant will also vary. Many of the general trends seen in the MSRE were shown to hold true with the presented work. These trends include: cover gas solubility, changes in interfacial area with system pressure and increases in xenon removal from mass transfer coefficients. 

Work as already began on coupling the fifth piece of modeling required for multi-physics simulations shown back in Figure \ref{fig:multiphysicsMSR}. This piece is the thermochemical state of the system. In this work the thermochemistry library Thermochimica \cite{piro2013} is used to calculate equilibrium composition and phase distribution. The library will be used to calculate and update the equilibrium FP speciation to be used as the driving force for species transport. These FPs include gaseous, reduced noble metals and surface corrosion products. 